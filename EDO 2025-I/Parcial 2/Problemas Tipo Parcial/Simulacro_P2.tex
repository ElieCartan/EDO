
\begin{enumerate}
    \item Se sabe que la solución del PVI
    $$\begin{cases}
        t^2y'' + ty'- 9y = 0 & t >0 \\
        y(1) = 4; & y'(1)=A
    \end{cases}$$
    satisface $$\lim_{t \to \infty} y(t) = 0$$
    Encuentre el valor de $A$.
    \item Se sabe que la solución del PVI
    $$\begin{cases}
        t^2y'' + ty' -9y = 0 & t >0 \\
        y(1) = A; & y'(1)=3
    \end{cases}$$
    satisface $$\lim_{t \to \infty} y(t) = 0$$
    Encuentre el valor de $A$.
    \item La solución general de una ecuación diferencial homogénea con coeficientes constanes de orden 5 es
    $$y= c_1 + c_2x + c_3x^2+c_4e^{-5x}+c_5e^{5x}$$
    entonces la ecuación diferencial es
    $$y^{(5)}+ Ay^{(3)} + By' = 0$$
    \begin{itemize}
        \item[a.] Encuentre el valor de $A$.
        \item[b.] Encuentre el valor de $B$.
    \end{itemize}
    
    \item La solución general de una ecuación diferencial homogénea con coeficientes constantes de orden 6 es
$$y = c_1 + c_2x + c_3e^{2x} + c_4e^{-2x} + c_5\cos(3x) + c_6\sin(3x)$$
entonces la ecuación diferencial es
$$y^{(6)} + Ay^{(4)} + By^{(2)} + Cy = 0$$
\begin{itemize}
    \item[a.] Encuentre el valor de $A$.
    \item[b.] Encuentre el valor de $B$.
    \item[c.] Encuentre el valor de $C$.
\end{itemize}
\item La solución general de una ecuación diferencial homogénea con coeficientes constantes de orden 5 es
$$y = c_1e^{3x} + c_2e^{-x} + c_3xe^{-x} + c_4\cos(2x) + c_5\sin(2x)$$
entonces la ecuación diferencial es
$$y^{(5)} + Ay^{(4)} + By^{(3)} + Cy^{(2)} + Dy' + Ey = 0$$
\begin{itemize}
    \item[a.] Encuentre el valor de $A$.
    \item[b.] Encuentre el valor de $B$.
    \item[c.] Encuentre el valor de $C$.
    \item[d.] Encuentre el valor de $D$.
    \item[e.] Encuentre el valor de $E$.
\end{itemize}

\item Considere la ecuación diferencial
$$y'' -10y' +25y = \dfrac{e^{5x}}{x}; \quad x>0$$
La solución general de dicha ecuación es de la forma 
$$y = y_h + y_p$$
donde $y_h = c_1e^{Ax} + c_2xe^{Ax}$, es la solución general de la ecuación diferencial homogénea asociada, y 
$$y_p = u_1e^{Ax} + u_2xe^{Ax}$$
es una solución particular de la ecuación original, donde
$$u_1 = Bx; \quad u_2 = C \ln(x)$$
\begin{itemize}
    \item Encuentre el valor de $A$.
    \item Encuentre el valor de $B$.
    \item Encuentre el valor de $C$.
\end{itemize}


\item Considere la ecuación diferencial
$$y'' - 6y' + 9y = \frac{e^{3x}}{x^2}; \quad x > 0$$
La solución general de dicha ecuación es de la forma
$$y = y_h + y_p$$
donde $y_h = c_1e^{Ax} + c_2xe^{Ax}$, es la solución general de la ecuación diferencial homogénea asociada, y
$$y_p = u_1e^{Ax} + u_2xe^{Ax}$$
es una solución particular de la ecuación original, donde
$$u_1 = \frac{B}{x}; \quad u_2 = C\ln(x)$$
\begin{itemize}
    \item Encuentre el valor de $A$.
    \item Encuentre el valor de $B$.
    \item Encuentre el valor de $C$.
\end{itemize}

\item Considere la ecuación diferencial
$$y'' + 4y' + 4y = \frac{e^{-2x}}{x^3}; \quad x > 0$$
La solución general de dicha ecuación es de la forma
$$y = y_h + y_p$$
donde $y_h = c_1e^{Ax} + c_2xe^{Ax}$, es la solución general de la ecuación diferencial homogénea asociada, y
$$y_p = u_1e^{Ax} + u_2xe^{Ax}$$
es una solución particular de la ecuación original, donde
$$u_1 = \frac{B}{x^2}; \quad u_2 = C\left(\ln(x) - \frac{1}{x}\right)$$
\begin{itemize}
    \item Encuentre el valor de $A$.
    \item Encuentre el valor de $B$.
    \item Encuentre el valor de $C$.
\end{itemize}

\item Para la ecuación diferencial
$$y'' + 2y' - 8y = 4e^{-4x} + 6xe^{2x} + 9\cos(3x)$$
donde la solución homogénea es $y_h = c_1e^{2x} + c_2e^{-4x}$, determine la forma correcta de la solución particular $y_p$.

La forma de la solución particular es:
$$y_p = Axe^{-4x} + (Bx + C)xe^{2x} + D\cos(3x) + E\sin(3x)$$

Después de aplicar el método de coeficientes indeterminados y resolver el sistema resultante, se encuentra que:
$$A = F, \quad B = G, \quad C = H, \quad D = I, \quad E = J$$
\begin{itemize}
    \item[a.] Encuentre el valor de $F$.
    \item[b.] Encuentre el valor de $G$.
    \item[c.] Encuentre el valor de $H$.
    \item[d.] Encuentre el valor de $I$.
    \item[e.] Encuentre el valor de $J$.
\end{itemize}

\item Considere la ecuación diferencial
$$y'' + y' - 2y = 24e^{x} + 25\sin(2x) - 20\cos(2x)$$
La solución general es $y = y_h + y_p$ donde $y_h = c_1e^{x} + c_2e^{-2x}$ y
$$y_p = Axe^{x} + B\sin(2x) + C\cos(2x)$$

\begin{itemize}
    \item[a.] Encuentre el valor de $A$.
    \item[b.] Encuentre el valor de $B$.
    \item[c.] Encuentre el valor de $C$.
\end{itemize}

\item Considere la ecuación diferencial
$$y'' - 6y' + 9y = 54xe^{3x} + 36x^2 - 72x + 45$$
La solución general es $y = y_h + y_p$ donde $y_h = c_1e^{3x} + c_2xe^{3x}$ y
$$y_p = Ax^2e^{3x} + Bx^2 + Cx + D$$


\begin{itemize}
    \item[a.] Encuentre el valor de $A$.
    \item[b.] Encuentre el valor de $B$.
    \item[c.] Encuentre el valor de $C$.
    \item[d.] Encuentre el valor de $D$.
\end{itemize}

Considere la ecuación diferencial
$$y'' + 4y = 20e^{-x} + 12x\cos(2x) + 16x\sin(2x)$$
La solución general de dicha ecuación es de la forma
$$y = y_h + y_p$$
donde $y_h = c_1\cos(2x) + c_2\sin(2x)$ es la solución general de la ecuación diferencial homogénea asociada, y la solución particular tiene la forma
$$y_p = Ae^{-x} + (Bx + C)\cos(2x) + (Dx + E)\sin(2x)$$

\begin{itemize}
    \item[a.] Encuentre el valor de $A$.
    \item[b.] Encuentre el valor de $B$.
    \item[c.] Encuentre el valor de $C$.
    \item[d.] Encuentre el valor de $D$.
    \item[e.] Encuentre el valor de $E$.
\end{itemize}

\item Considere la ecuación diferencial de Cauchy-Euler
$$x^2y'' - 3xy' + 4y = 8x^2 + 12x^3\ln(x); \quad x > 0$$
La solución general de dicha ecuación es de la forma
$$y = y_h + y_p$$
donde $y_h = c_1x^2 + c_2x^2\ln(x)$ es la solución general de la ecuación diferencial homogénea asociada, y la solución particular tiene la forma
$$y_p = Ax^2\ln^2(x) + Bx^3 + Cx^3\ln(x)$$

\begin{itemize}
    \item[a.] Encuentre el valor de $A$.
    \item[b.] Encuentre el valor de $B$.
    \item[c.] Encuentre el valor de $C$.
\end{itemize}

\item Considere la ecuación diferencial de Cauchy-Euler
$$x^2y'' + xy' - y = 6x + 8x^{-1} + 15x^2; \quad x > 0$$
La solución general de dicha ecuación es de la forma
$$y = y_h + y_p$$
donde $y_h = c_1x + c_2x^{-1}$ es la solución general de la ecuación diferencial homogénea asociada, y la solución particular tiene la forma
$$y_p = Ax^2 + Bx\ln(x) + Cx^{-1}\ln(x)$$

\begin{itemize}
    \item[a.] Encuentre el valor de $A$.
    \item[b.] Encuentre el valor de $B$.
    \item[c.] Encuentre el valor de $C$.
\end{itemize}

\item Considere la ecuación diferencial de Cauchy-Euler
$$x^2y'' - xy' + y = 4x + 9x\cos(\ln x) + 12x\sin(\ln x); \quad x > 0$$
La solución general de dicha ecuación es de la forma
$$y = y_h + y_p$$
donde $y_h = c_1x + c_2x\ln(x)$ es la solución general de la ecuación diferencial homogénea asociada, y la solución particular tiene la forma
$$y_p = Ax\ln^2(x) + Bx\ln(x)\cos(\ln x) + Cx\ln(x)\sin(\ln x)$$

\begin{itemize}
    \item[a.] Encuentre el valor de $A$.
    \item[b.] Encuentre el valor de $B$.
    \item[c.] Encuentre el valor de $C$.
\end{itemize}

\item  Considere la ecuación diferencial de Cauchy-Euler
$$x^2y'' - xy' + y = 6x + 18x\cos(\ln x) + 16x\sin(\ln x); \quad x > 0$$
La solución general es $y = y_h + y_p$ donde $y_h = c_1x + c_2x\ln(x)$ y
$$y_p = Ax\ln^2(x) + Bx\ln(x)\cos(\ln x) + Cx\ln(x)\sin(\ln x)$$


\begin{itemize}
    \item[a.] Encuentre el valor de $A$.
    \item[b.] Encuentre el valor de $B$.
    \item[c.] Encuentre el valor de $C$.
\end{itemize}

\item La solución general de una ecuación diferencial lineal homogénea con coeficientes constantes de orden 3 es
$$y = c_1e^{-2x} + c_2e^{3x} + c_3xe^{3x}$$
entonces la ecuación diferencial es
$$y''' + Ay'' + By' + Cy = 0$$

\begin{itemize}
    \item[a.] Encuentre el valor de $A$.
    \item[b.] Encuentre el valor de $B$.
    \item[c.] Encuentre el valor de $C$.
\end{itemize}

\item La solución general de una ecuación diferencial lineal homogénea con coeficientes constantes de orden 4 es
$$y = c_1e^{x} + c_2e^{-x} + c_3\cos(2x) + c_4\sin(2x)$$
entonces la ecuación diferencial es
$$y^{(4)} + Ay''' + By'' + Cy' + Dy = 0$$

\begin{itemize}
    \item[a.] Encuentre el valor de $A$.
    \item[b.] Encuentre el valor de $B$.
    \item[c.] Encuentre el valor de $C$.
    \item[d.] Encuentre el valor de $D$.
\end{itemize}

\item Considere la ecuación diferencial con coeficientes constantes
$$y'' - 5y' + 6y = 0$$
con condiciones iniciales $y(0) = 7$ y $y'(0) = 13$.
La solución general es $y = c_1e^{2x} + c_2e^{3x}$

\begin{itemize}
    \item[a.] Encuentre el valor de $c_1$.
    \item[b.] Encuentre el valor de $c_2$.
\end{itemize}

\item La solución general de una ecuación diferencial lineal homogénea con coeficientes constantes de orden 5 es
$$y = c_1 + c_2x + c_3x^2 + c_4e^{4x} + c_5e^{-3x}$$
entonces la ecuación diferencial es
$$y^{(5)} + Ay^{(4)} + By^{(3)} + Cy'' + Dy' + Ey = 0$$

\begin{itemize}
    \item[a.] Encuentre el valor de $A$.
    \item[b.] Encuentre el valor de $B$.
    \item[c.] Encuentre el valor de $C$.
    \item[d.] Encuentre el valor de $D$.
    \item[e.] Encuentre el valor de $E$.
\end{itemize}

\item Se sabe que la solución del PVI
$$\begin{cases}
    t^2y'' - 3ty' + 4y = 0 & t > 0 \\
    y(1) = 6; & y'(1) = A
\end{cases}$$
satisface $$\lim_{t \to \infty} y(t) = 0$$
Encuentre el valor de $A$.

\item  Se sabe que la solución del PVI
$$\begin{cases}
    t^2y'' + ty' - 16y = 0 & t > 0 \\
    y(1) = 10; & y'(1) = A
\end{cases}$$
satisface $$\lim_{t \to \infty} y(t) = 0$$
Encuentre el valor de $A$.

\item Se sabe que la solución del PVI
$$\begin{cases}
    t^2y'' - 5ty' + 6y = 0 & t > 0 \\
    y(2) = 8; & y'(2) = A
\end{cases}$$
satisface $$\lim_{t \to \infty} y(t) = 0$$
Encuentre el valor de $A$.

\item Se sabe que la solución del PVI
$$\begin{cases}
    t^2y'' + 3ty' - 8y = 0 & t > 0 \\
    y(1) = 12; & y'(1) = A
\end{cases}$$
satisface $$\lim_{t \to \infty} y(t) = 0$$
Encuentre el valor de $A$.

\item Se sabe que la solución del PVI
$$\begin{cases}
    t^2y'' - 2ty' - 15y = 0 & t > 0 \\
    y(1) = 12; & y'(1) = A
\end{cases}$$
satisface $$\lim_{t \to \infty} y(t) = 0$$
Encuentre el valor de $A$.

\item Se sabe que la solución del PVI
$$\begin{cases}
    t^2y'' + 3ty' - 4y = 0 & t > 0 \\
    y(1) = 8; & y'(1) = A
\end{cases}$$
satisface $$\lim_{t \to \infty} y(t) = 0$$
Encuentre el valor de $A$.

\item La posición de $x(t)$ de un objeto conectado a un resorte y ejecutando un movimiento rectilíneao está dado por la ecuación diferencial
$$x'' + \beta x' +4x = F(t)$$
donde $F(t)$ es una fuerza externa aplicada.
\begin{itemize}
    \item[a.] Si $F(t) = 0, \beta = A$. Encuentre el valor de $A$ para que el movimiento sea críticamente amortigüado.
    \item[b.] Si $\beta = 0, F(t) = 5\sin(Bt)$, encuentre el valor de $B$ para que se produzca resonancia.
    \item[c.] Si $\beta = 4, F(t) = 12$. Encuentre $$C = \lim_{t\to \infty} x(t)$$
\end{itemize}

\item Una masa que pesa 128 libras alarga $\frac{2}{25}$ pies un resorte. Al inicio la masa se libera desde un punto que está 20 pies arriba de la posición de equilibrio con una velocidad de 3 pies/segundo hacia arriba. El problema de valor inicial que describe la posición $x(t)$ de la masa en el tiempo $t$ es 
$$\begin{cases}
    x'' + Ax' + Bx = 0; & t \geq 0 \\
    x(0) = C; & x'(0) = D
\end{cases}$$
\begin{itemize}
    \item[a.] Encuentre el valor de $A$.
    \item[b.] Encuentre el valor de $B$.
    \item[c.] Encuentre el valor de $C$.
    \item[d.] Encuentre el valor de $D$.
    \item[e.] Si los cíclos completos que habrá realizado la masa al final de $5 \pi$ segundos es $E$, encuentre el valor de $E$. 
\end{itemize}
\item Una partícula describe un movimiento armónico simple de manera que su aceleración máxima es $34^2 \cdot 18\ {\rm metros}/{\rm segundo}^2$ y su velocidad máxima es $34 \cdot 18\ {\rm metros}/{\rm segundo}$. \\
Si la amplitud $A$ y la frecuencia natural de oscilación $\omega = \sqrt{\frac{k}{m}}$ es $B$, entonces
\begin{itemize}
    \item[a.] Encuentre el valor de $A$.
    \item[b.] Encuentre el valor de $B$. 
\end{itemize}

\item Una masa que pesa 96 libras alarga $\frac{3}{32}$ pies un resorte. Al inicio la masa se libera desde un punto que está 15 pies abajo de la posición de equilibrio con una velocidad de 4 pies/segundo hacia abajo. El problema de valor inicial que describe la posición $x(t)$ de la masa en el tiempo $t$ es 
$$\begin{cases}     
x'' + Ax' + Bx = 0; & t \geq 0 \\     
x(0) = C; & x'(0) = D 
\end{cases}$$

\begin{itemize}
    \item Encuentre el valor de $A$.
    \item Encuentre el valor de $B$.
    \item Encuentre el valor de $C$.
    \item Encuentre el valor de $D$.
    \item Si los ciclos completos que habrá realizado la masa al final de $4\pi$ segundos es $E$, encuentre el valor de $E$.
\end{itemize}

\item Una masa que pesa 64 libras alarga $\frac{1}{16}$ pies un resorte. Al inicio la masa se libera desde un punto que está 12 pies arriba de la posición de equilibrio con una velocidad de 2 pies/segundo hacia abajo. El problema de valor inicial que describe la posición $x(t)$ de la masa en el tiempo $t$ es 
$$\begin{cases}     
x'' + Ax' + Bx = 0; & t \geq 0 \\     
x(0) = C; & x'(0) = D 
\end{cases}$$

\begin{itemize}
    \item Encuentre el valor de $A$.
    \item Encuentre el valor de $B$.
    \item Encuentre el valor de $C$.
    \item Encuentre el valor de $D$.
    \item Si los ciclos completos que habrá realizado la masa al final de $6\pi$ segundos es $E$, encuentre el valor de $E$.
\end{itemize}

\item La posición de $x(t)$ de un objeto conectado a un resorte y ejecutando un movimiento rectilíneo está dado por la ecuación diferencial
$$x'' + \beta x' + 25x = F(t)$$
donde $F(t)$ es una fuerza externa aplicada.

\begin{itemize}
    \item Si $F(t) = 0, \beta = A$. Encuentre el valor de $A$ para que el movimiento sea críticamente amortiguado.
    \item Si $\beta = 0, F(t) = 7\sin(Bt)$, encuentre el valor de $B$ para que se produzca resonancia.
    \item Si $\beta = 10, F(t) = 50$. Encuentre $$C = \lim_{t\to \infty} x(t)$$
\end{itemize}

\item La posición de $x(t)$ de un objeto conectado a un resorte y ejecutando un movimiento rectilíneo está dado por la ecuación diferencial
$$x'' + \beta x' + 16x = F(t)$$
donde $F(t)$ es una fuerza externa aplicada.

\begin{itemize}
    \item Si $F(t) = 0, \beta = A$. Encuentre el valor de $A$ para que el movimiento sea críticamente amortiguado.
    \item Si $\beta = 0, F(t) = 3\sin(Bt)$, encuentre el valor de $B$ para que se produzca resonancia.
    \item Si $\beta = 8, F(t) = 32$. Encuentre $$C = \lim_{t\to \infty} x(t)$$
\end{itemize}
\item La posición de $x(t)$ de un objeto conectado a un resorte y ejecutando un movimiento rectilíneo está dado por la ecuación diferencial
$$x'' + \beta x' + 9x = F(t)$$
donde $F(t)$ es una fuerza externa aplicada.

\begin{itemize}
    \item Si $F(t) = 0, \beta = A$. Encuentre el valor de $A$ para que el movimiento sea críticamente amortiguado.
    \item Si $\beta = 0, F(t) = 8\sin(Bt)$, encuentre el valor de $B$ para que se produzca resonancia.
    \item Si $\beta = 6, F(t) = 18$. Encuentre $$C = \lim_{t\to \infty} x(t)$$
\end{itemize}
\item Si el wronskiano $W$ de $f$ y $g$ es $5e^{6t}$ y si $f(t) = e^{3t}$, encuentre $g(t)$.

\item Si el wronskiano $W$ de $f$ y $g$ es $-2e^{-2t}$ y si $f(t) = e^{-t}$, encuentre $g(t)$.

\item Si el wronskiano $W$ de $f$ y $g$ es $4e^{8t}$ y si $f(t) = e^{4t}$, encuentre $g(t)$.

\item Si $y_1$ e $y_2$ son soluciones linealmente independientes (L.I.) de la ecuación diferencial $t^2 y'' + ty' + (t^3-1)y = 0$ y si el wronskiano de $y_1$ e $y_2$ evaluado en $2$ es $W(y_1, y_2)(2) = 3$, halle $W(y_1, y_2)(6)$.

\item Si $y_1$ e $y_2$ son soluciones L.I. de la ecuación diferencial $y'' - 2xy' + \cos(x) y = 0$ (donde $y$ es función de $x$) y si el wronskiano de $y_1$ e $y_2$ evaluado en $0$ es $W(y_1, y_2)(0) = 1$, halle $W(y_1, y_2)(2)$.

\item Si $y_1$ e $y_2$ son soluciones L.I. de la ecuación diferencial $(1-x^2)y'' - 2xy' + \sin(x) y = 0$ (donde $y$ es función de $x$) y si el wronskiano de $y_1$ e $y_2$ evaluado en $0$ es $W(y_1, y_2)(0) = 4$, halle $W(y_1, y_2)(0.5)$. Asuma que $x \in (-1,1)$.

\item Considere el PVI 
$$\begin{cases}
    \frac{1}{2}y'' - y = 0; \\
    y(0) = 3, & y'(0) = \beta
\end{cases}$$
El valor de $\beta$ de modo que la solución del PVI tienda a cero es $\beta = $.

\item La solución general de una ecuación diferencial lineal  homogénea de Cauchy-Euler de orden 2 es
$$y= c_1x^2\cos(3\ln(x)) + c_2x^2\sin(3\ln(x))$$
Entonces la ecuación diferencial es
$$x^2y'' + Axy' + By = 0$$
\begin{itemize}
    \item[a.] El valor de $A$ es:
    \item[b.] El valor de $B$ es: 
\end{itemize}

\item Una masa que pesa $32 \cdot 2$ libras estira un resorte $\frac{g}{9}$ pies. Suponga que la masa está sujeta a un amortiguador viscoso cuya constante de amortigüamiento $\beta$ es de 12 ${\rm libras}\cdot {\rm segundo}/{\rm pie}$. Desde la posición de equilibrio se jala la masa 1 pie hacia abajo y se suelta imprimiéndole una velocidad de 6 pies/segundo hacia arriba. 
\begin{itemize}
    \item[a.] El PVI que describe la posición $x(t)$ de la masa en el tiempo $t$ es:
    $$\begin{cases}
        x'' +bx' + cx = 0; \\
        x(0) = d, & x'(0)= e
    \end{cases}$$
    \item[b.] La posición $x(t)$ de la masa en cualquier instante $t\geq 0$ es 
    $$x(t) = fe^{gt} +hte^{gt}$$
    de el valor de $f,g$ y $h$.
    \item[c.] La masa pasa por la posición de equilibrio en el tiempo $t = $
\end{itemize}

\item La posición de un cuerpo cuya masa, que está sujeta a un resorte está dada por
$$x(t) = -3\cos(4t)-3\sin(4t)$$
pies, para cada instante de tiempo $t\geq 0$ segundos.
\begin{itemize}
    \item[a.] Al escribir la posición en la forma
    $$x(t) = A\sin(4t+ \Phi)$$
    De los valores de $A$ y $\Phi$.
    \item[b.] Los cíclos completos que habrá realizado la masa al final de $15 \pi$ segundos son:
    \item[c.] El instante de tiempo en que la masa pasa por la posición de equilibrio pro primera vez es $t = A\frac{\pi}{16}$ segundos. De el valor de $A$.
\end{itemize}

\item Considere la ecuación 
$$y'' +25y = 12 \cdot 25 \sec(5x); \quad -\frac{\pi}{2}< x<\frac{\pi}{2}$$
\begin{itemize}
    \item[a.] Halle el conjunto $S = \{y_1,y_2\}$ que sea un CFS para la EDO homogénea asociada tal que  se cumpla
    $$y_1(0) = 3, \quad y_2'(0) = 20$$
    \item[b.] Si $S= \{y_1,y_2\}$ es el CFS de la EDO homogénea asociada hallado en el inciso anterior, encuentre el valor de $B = W(y_1,y_2)(x)$.
    \item[c.] Una solución particular $y_p$ tiene la forma
    $$y_p = u_1y_1 + u_2y_2$$
    donde $$u_1 = C\ln(\cos(5x)); \quad u_2 = Dx$$ halle los valores de $C$ y $D$.
\end{itemize}
\end{enumerate}

